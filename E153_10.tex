\documentclass[12pt,letterpaper]{report}
\usepackage[margin=1in]{geometry}
\usepackage{graphicx}
\usepackage{amsmath}
\usepackage[font=small,labelfont=bf]{caption}
\usepackage[justification=centering]{caption}
\usepackage{tikz}
\usepackage{circuitikz}
\usepackage{siunitx}
\usepackage{float}
\newlength \figwidth
\setlength \figwidth {0.75\linewidth}

\begin{document}

\title{E153 Laboratory Assignment \#10}
\author{Courtney Keeler and Stephen Pinto\\
Harvey Mudd College}
\date{December 13, 2013}
\maketitle

\section*{List of Materials}
\begin{itemize}
	\item Agilent E360A power supply (x3)
	\item Hewlett Packard 33120A Function Generator
	\item Tektronix 2212 Oscilloscope
	\item Pomona 4550B (10X probe)
	\item 2N2222A transistor
	\item 2N3440 transistor
	\item Standard resistors
	\item Standard capacitors
\end{itemize}

\section*{Purpose}
The purpose of this lab is to build the cascode amplifier designed in Design Project \#2 and verify the performance specifications in lab. The final design of the cascode amplifier is shown in Figure %\ref{fig:cascode_circuit}
.

\section*{Results}

In order to have both a +80 and +12 volt DC power supply for the circuit, a voltage divider must be constructed to drop an 80 V input voltage to 12 volts. The circuit for such a task is shown in Figure %\ref{fig:voltage_divider}
.


The resistor, capacitor, and inductor values used are listed in Table \ref{tab:values}.
\begin{table}[H]
\centering
\begin{tabular}{|c|c|c|}
	\hline
	 & Designed Value & Measured Value \\
	\hline
	$R_1$ & 33 k$\Omega$ & 32.65 k$\Omega$\\
	\hline
	$R_2$ & 10 k$\Omega$ & 9.88 k$\Omega$ \\
	\hline
	$R_E$ & 82 $\Omega$ & 82.6 $\Omega$ \\
	\hline
	$R_C$ & 2.2 k$\Omega$ & 2.151 k$\Omega$ \\
	\hline
	$C_1$ & 10 $\mu$F & 10.1 $\mu$F \\
	\hline
	$L_1$ & 68 $\mu$H & 82 $\mu$H \\
	\hline
\end{tabular}
\caption{Table of designed versus measured values for the components of the cascode amplifier.}
\label{tab:values}
\end{table}

\subsection*{DC Offset}
With no AC input coming in, the DC offset at the output terminal was 49.06 V. With a target oscillation of $\pm 25$V, a DC offset of about 49 V will make an output signal of 50 $V_{pp}$ oscillate between 24V and 74V. The specifications give a suggested range of 20V to 70V but requires the 50 $V_{pp}$ output for a 2 $V_{pp}$ input. Assuming the gain is correct and that requirement is met (see next section), we think our DC offset is acceptable.

\subsection*{Gain}

% See pic

\subsection*{Bandwidth}

% See pic
% 3.65 Hz to 6.45 MHz

\subsection*{Input Resistance}

% See pic
% Rin = 1 / (2*pi*fc*c1) = 4317 Ohm

\section*{Conclusion}

In conclusion, this lab has shown that the results achieved in Multisim do not exactly match those seen in the laboratory setting. There are many subleties to designed such a high amplification circuit that are easily overlooked in simulation (such as resistor power limits). The purpose of the lab, which was to build a working cascode amplifier circuit, was met.

\end{document}
