
\documentclass[12pt,letterpaper]{report}
\usepackage[margin=1in]{geometry}
\usepackage{graphicx}
\usepackage{amsmath}

% info for header block in upper right hand corner
%\name{Courtney Keeler and Stephen Pinto}
%\class{E153}
%\assignment{Lab \#1}
%\duedate{September 30, 2013}

\begin{document}

Tektronix 2212 Oscilloscope\\
Hewlett Packard 33120A Function generator\\
Pomona 4550B (10X probe)

\begin{center}
Presentation of sine waves\\
    \begin{tabular}{| c | c | c |}
    \hline
    
    frequency & analog & digital \\
    \hline
    1 Hz & * & 283.1 m$\text{V}_{\text{pp}}$, 991.5 mHz \\
    1 kHz & 1.012 $\text{V}_{\text{pp}}$, 1.025 kHz & 1.003 $\text{V}_{\text{pp}}$, 1.002 kHz \\
    1 MHz & 1.003 $\text{V}_{\text{pp}}$, 1.003 MHz & 991.7 m$\text{V}_{\text{pp}}$, 1.00 MHz \\
    10 MHz & 976.2mV, 10.06 MHz & ** \\
    
    \hline
    \end{tabular}
\end{center}
* Can see the sin wave moving across the screen but cannot capture samples and therefore cannot measure signal
\\[.1in]
** Can't be measured since time scale minimum is .4us/div while storing is on
\\[.3in]
Using the multiplier on the clock allows for a sec/div of 4ns
\\[.3in]

\begin{center}

Capacitance\\
	\begin{tabular}{| c | c |}
	\hline
	
	BNC/Coax & 10X Probe \\
	\hline
	108 pF & 18 pF \\

	\hline
	\end{tabular}
\end{center}

\begin{center}
Resistive Loading\\
	\begin{tabular}{| c | c |}
	\hline
	
	BNC/Coax & 10X Probe \\
	\hline
	1 M$\Omega $ & 10 M$\Omega$ \\

	\hline
	\end{tabular}
\end{center}

\begin{center}
Reactive Loading\\
	\begin{tabular}{| c | c | c |}
	\hline
	
	Frequency & BNC & 10X Probe \\
	\hline
	100 Hz & 78.1 M$\Omega$ & 690 M$\Omega$\\
	1000 Hz & 7.81 M$\Omega$ & 69 M$\Omega$\\
	1 MHz & 7.81 k$\Omega$ & 69 k$\Omega$\\
	
	\hline
	\end{tabular}
\end{center}

For the 10X probe:
$$\text{C}_{\text{total}} = 14.4 \text{pF}$$
$$ \frac{1}{\omega \text{C}_{\text{total}}} = 10\text{M}\Omega$$
$$ \omega = 6.94 \text{kHz} $$

%
%Measure the capacitance of a BNC/Coax probe: .108nF (.011nF is the splitter alone, multimeter always shows .004 or .005 nF)
%
%Measure the input capacitance of the 10X probe (Pomona 4550B): 
%$C_{probe}$ = .013 nF
%$C_{equiv}$ = .021nF
%Sanity check: $C_{equiv}$ should work out to be $C_{probe}$ in series with $C_{BNC}$ (where $C_{BNC}$ = 0.053nF). 
%
%Nevermind, we ignore the capacitance due to the length of cable attached to the 10X probe.
%
% Experimental corner frequency for the BNC cable: 135 kHz (this is the frequency where the amplitude is .707)
% Multisim: 122.7 kHz difference: BNC splitter introduces another capacitor, multisim model doesn't include that or the resistance from the cables
% we measure the BNC splitter capacitance to verify: 12pF 
% Square wave: back to 1Vpp (909 mV). 10x corner: 164.2 mV, turned a square wave into a triangle wave
% Triangle: 678 mV (frequency limited (100kHz), can't scope it at corner frequency)
%
% Experimental corner frequency for the 10X probe:  (starting at 100mV) 1.15MHz
% Multisim: difference:
% square wave: 91.67 mV. 10x corner: 16.42 mV
% triangle wave: frequency limited
%

\end{document}