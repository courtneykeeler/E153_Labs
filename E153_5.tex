\documentclass[12pt,letterpaper]{report}
\usepackage[margin=1in]{geometry}
\usepackage{graphicx}
\usepackage{amsmath}
\usepackage[font=small,labelfont=bf]{caption}
\usepackage[justification=centering]{caption}
\usepackage{tikz}
\usepackage{circuitikz}
\usepackage{siunitx}
\usepackage{float}
\newlength \figwidth
\setlength \figwidth {0.75\linewidth}

\begin{document}

\title{E153 Laboratory Assignment \#5}
\author{Courtney Keeler and Stephen Pinto\\
Harvey Mudd College}
\date{October 21, 2013}
\maketitle

\section*{List of Materials}
\begin{itemize}
	\item Tektronix 2212 Oscilloscope
	\item Pomona 4550B (10X probe)
	\item Elenco LCM-1950 Multimeter
	\item 2N3904 transistor
	\item Lab Station 7 Transformer (s.n. 139 01 665)
	\item Simpson 260 (s.n. 139 01 156)
	\item Model EUW-28 Resistance Substitution Box (s. n. 77901165)
	\item 10 kOhm potentiometer
\end{itemize}

\section*{Purpose}
The purpose of this lab is to build bi-polar transistor circuits and learn about the properties of transistors through their unique uses.

\section*{4.1 Transistor Junctions}
\subsection*{Procedure}

\begin{enumerate}
\item Obtain a 2N3904 transistor and measure the voltage across the the BC junction using the diode test function
\item Repeat the previous step for the BE junction
\end{enumerate}

\subsection*{Results}

\begin{table}[ht]
\caption{Voltage across each "diode"} % title of Table
\centering 
    \begin{tabular}{| c | c |} 
    \hline
    $V_{\text{BC}}$ & 0.711 V \\
    $V_{\text{BE}}$ & 0.717 V \\
    \hline
    \end{tabular}
    \label{table:section_1}
\end{table}

\subsection*{Analysis}

We are looking to see if the transistor is in working condition. If it is, the two junctions BE and BC will look like back to back forward biased diodes when testing them with the digital multimeter's diode test function. As shown in Table \ref{table:section_1}, each junction has a drop of 0.7 V, just like a forward biased diode - meaning the transistor is working.

\section*{4.5 Transistor Current Gain}
\subsection*{Procedure}

\begin{enumerate}
\item Construct the circuit shown in Figure \ref{fig:4.5_circuit}
\item Start with an initial R (from the resistance substitution box) of 4.7 MOhm
\item Measure $I_C$ and $I_B$ using two digital multimeters
\item Turn on both power supplies and record the currents
\item Repeat for enough values of R to cover three decades of collector current.
\end{enumerate}

\begin{figure}[H]
\centering
\includegraphics[width=\figwidth, keepaspectratio=true]{lab5/circuit_2.jpg}
\caption{Circuit to measure the gain of the transistor.}
\label{fig:4.5_circuit}
\end{figure}

\subsection*{Results}
Note that 
$$
\beta = \frac{I_C}{I_B}
$$
and
$$
I_B = \frac{V_{R_B}}{R_B}.
$$

\begin{table}[ht]
\caption{Experimental Results for part 4.5} % title of Table
\centering 
    \begin{tabular}{| c | c | c | c | c | c |}
    \hline  
    $R$ & $R_B$ & $I_C$ & $V_{R_B}$ & $I_B$ & $\beta$\\
    \hline
    10 M$\Omega$  & 4.7 k$\Omega$ & 0.09 mA & 2.0 mV  & 0.43 uA & 212 \\
    470 k$\Omega$ & 4.7 k$\Omega$ & 1.7 mA & 39.8 mV  & 8.47 uA & 201 \\
    100 k$\Omega$ & 4.7 k$\Omega$ & 8.1 mA & 197.7 mV & 42.1 uA & 193 \\
    \hline
    \end{tabular}
    \label{table:4_5_results}
\end{table}

\section*{4.6 Current Source}
\subsection*{Procedure}

\begin{enumerate}
\item Construct the current sink circuit shown in Figure \ref{fig:4.6_circuit}
\item Slowly vary the 1 kOhm potentiometer while looking for changes in the measured collector current (use a digital ammeter).
\item Monitor the $V_{CE}$ using a digital voltmeter
\item Observe what happens at the maximum resistance of the pot
\end{enumerate}

\begin{figure}[H]
\centering
\includegraphics[width=\figwidth, keepaspectratio=true]{lab5/circuit.jpg}
\caption{Current sink circuit.}
\label{fig:4.6_circuit}
\end{figure}

\subsection*{Results}
Since the slope of the transistor's $I_C$ vs. $V_{CE}$ curve is so shallow, it was difficult to accurately measure changes in $I_C$ as the potentiometer varied. The only two datapoints collected are summarized in Table \ref{table:4-6_results}

\begin{table}[ht]
\caption{Experimental Results for part 4.6} % title of Table
\centering 
    \begin{tabular}{| c | c |}
    \hline  
    $I_C$ & $V_{CE}$\\
    \hline
    5.29 mA & 2.85 V\\
    5.30 mA & 6.53 V\\
    \hline
    \end{tabular}
    \label{table:4-6_results}
\end{table}

\subsection*{Analysis}
The compliance range of this transistor is roughly 0.2 V upward (limited by the 15 V source of course.) This compliance range covers a change of about 0.02 mA in the collector current. Using the two data points in Table \ref{table:4-6_results}, the slope of the $I_C$ vs $V_{CE}$ curve is
$$
m = \frac{5.30 mA - 5.29 mA}{6.53V - 2.85V} = 2.7 \frac{\mu A}{V}
$$
Which makes the Early voltage
$$
V_A = \frac{5.29 mA}{2.7 \frac{\mu A}{V}} - 2.85 V = 1944 V
$$

This Early Voltage is ludicrously large and not believable. We attribute this poor result to an inability to record accurate changes in $I_C$ as we varied the potentiometer. The resolution of the ammeter was too poor to pick up the changes. We measured the two data points in table \ref{table:4-6_results} multiple times and the $V_{CE}$ values corresponding to $I_C$ = 5.29 and 5.30 mA changed every time. As such the slope of the compliance range might be hugely different than calculated above, which would change the Early Voltage.

\subsection*{Conclusion}

In conclusion, this lab has shown several properties of transistors. First, it has been shown that each branch (collector, emitter) of a transistor acts independently like a diode, with a .7 V drop in the forward direction. It has also been shown that even over three decades of collector current, the gain of a 2N3904 transistor does not vary significantly (from 193 to 212). The current sink circuit was intended to be used as a means to measure the early voltage of the transistor. However, it was discovered that the change in collector current was so small that the measurement devices were not precise enough to yield good data that lead to the calculation of a reasonable forward voltage. The purpose of the lab, which was to learn about the functioning of transistors in different conditions, was met.

\end{document}
