\documentclass[12pt,letterpaper]{report}
\usepackage[margin=1in]{geometry}
\usepackage{graphicx}
\usepackage{amsmath}
\usepackage[font=small,labelfont=bf]{caption}
\usepackage[justification=centering]{caption}
\usepackage{tikz}
\usepackage{circuitikz}
\usepackage{siunitx}
\newlength \figwidth
\setlength \figwidth {0.75\linewidth}

\begin{document}

\title{E153 Laboratory Assignment \#5}
\author{Courtney Keeler and Stephen Pinto\\
Harvey Mudd College}
\date{October 21, 2013}
\maketitle

\section*{List of Materials}
\begin{itemize}
	\item Tektronix 2212 Oscilloscope
	\item Pomona 4550B (10X probe)
	\item Elenco LCM-1950 Multimeter
	\item 2N3904 transistor
	\item Lab Station 7 Transformer (s.n. 139 01 665)
	\item Simpson 260 (s.n. 139 01 156)
	\item Model EUW-28 Resistance Substitution Box (s. n. 77901165)
	\item 10 kOhm potentiometer
\end{itemize}

\section*{Purpose}
The purpose of this lab is to build bi-polar transistor circuits and do neat things with them.

\section*{4.1 Transistor Junctions}
\subsection*{Procedure}

\begin{enumerate}
\item Obtain a 2N3904 transistor and measure the voltage across the the BC junction using the diode test function
\item Repeat the previous step for the BE junction
\end{enumerate}

\subsection*{Results}

BC: .711 V

BE: .717 V

\subsection*{Analysis}

We are looking to see if the individual junctions act as diodes when independent from one another. We also expect the BC junction to be the larger of the two.

\section*{4.5 Transistor Current Gain}
\subsection*{Procedure}

\begin{enumerate}
\item Construct the circuit shown in Figure %\ref{4.5_circuit.png}
\item Start with an initial R (from the resistance substitution box) of 4.7 MOhm
\item Add the current at Ic using the multimeter in series in the circuit
\item Turn on both power supplies and record the measured current
\item Repeat for many values of R
\end{enumerate}

\begin{enumerate}
\item Construct the circuit shown in Figure %\ref{4.5_circuit.png}
\item Use the multimeter to measure IB and the Simpson 260 to measure IC
\item Repeat for many values of R
\end{enumerate}

\subsection*{Results}
%R = 4.7 MOhm, Ic = .036 mA *, Vbase = 0.8mV 
%R = 1 MOhm, Ic = 0.15 mA *, Vbase = 3.6mV
%R = 470 kOhm, Ic = 0.29 mA
%R = 100 kOhm, Ic = 1.37 mA
%R = 47 kOhm, Ic = 2.50 mA *, Vbase = 60.1 mV
%R = 10 kOhm, Ic = 8.73 mA

R = 10 MOhm, Ic =0.9 mA , Vrb = 2.0 mV
R = 4.7 MOhm, Ic = .19 mA, Vrb = 4.5 mV
R = 470 kOhm, Ic = 1.7 mA, Vrb = 39.8 mV
R = 100 kOhm, Ic = 8.1 mA, Vrb = 197.7 mV

4.7 kOhm is really 4.58 kOhm

*measured with Simpson, not DVM

\subsection*{Analysis}


\subsection*{Calculations}

\section*{4.6 Current Source}
\subsection*{Procedure}

\begin{enumerate}
\item Construct the current sink circuit shown in Figure %\ref{3.4_circuit}
\item Slowly vary the 1 kOhm potentiometer while looking for changes in the measured collector current (use the Simpson).
\item Monitor the collector/emitter voltage using the mutlimeter
\item Observe what happens at the maximum resistance of the pot
\end{enumerate}

\subsection*{Results}

%0 kOhm--
%Vce: 13.85 V
%Ic: .89 mA
%
%1 kOhm--
%Vce = 12.94 V
%Ic = 0.89 mA
%
%0 kOhm--
%Vce =13.84  V
%Ic = 0.89mA
%
%10 kOhm--
%Vce =5.70  V
%Ic = 0.885mA

0 kOhm--
Vce = 10.28 V
Ic = 5.3 mA

Ic = 

Vce = 117.2 mV
Ic = 5.1 mA

Vce = 65.6  mV
Ic = 4.8 mA

Vce = 42.5 mV
Ic = 4.4 mA

Vce = 32.3 mV
Ic = 4.0 mA

Vce = 25.7 mV
Ic = 3.6 mA

Vce = 

Vce = 95.4 mV
Ic = 5.1 mA

Vce = .068 V
Ic=4.83

Vce = 20.7 mV
Ic = 3.2 mA
---------------- started measuring the voltage w/ Simpson and the current w/ digital multimeter

Vce = 10 V
Ic = 5.32 mA

Vce = 7.4 V
Ic = 5.30 mA

Vce = 2.8 V
Ic = 5.29 mA

Vce = 9.6 V
Ic = 5.31 mA
-------------- started measuring with two digital meters
Vce = 2.85 V
Ic = 5.29 mA

Vce = 6.53 V
Ic = 5,3 mA

\subsection*{Analysis}

Explain the max resistance behavior in terms of voltage compliance of the current source

What causes the variations in output current as the load is varied within the compliance range? Verify by explanation and by making the appropriate measurements

\subsection*{Calculations}


\end{document}
